\documentclass{anstrans}
%%%%%%%%%%%%%%%%%%%%%%%%%%%%%%%%%%%
\title{Benefits of Siting a Borehole Repository at a Non-operating Nuclear 
Facility}
\author{Jin Whan Bae }

\institute{
Dept. of Nuclear Plasma, and Radiological Engineering, University of Illinois at Urbana-Champaign
\and
Urbana, IL
}

\email{jbae11@illinois.edu}

%%%% packages and definitions (optional)
\usepackage{graphicx} % allows inclusion of graphics
\usepackage{booktabs} % nice rules (thick lines) for tables
\usepackage{microtype} % improves typography for PDF

\newcommand{\SN}{S$_N$}
\renewcommand{\vec}[1]{\bm{#1}} %vector is bold italic
\newcommand{\vd}{\bm{\cdot}} % slightly bold vector dot
\newcommand{\grad}{\vec{\nabla}} % gradient
\newcommand{\ud}{\mathop{}\!\mathrm{d}} % upright derivative symbol

\begin{document}
%%%%%%%%%%%%%%%%%%%%%%%%%%%%%%%%%%%%%%%%%%%%%%%%%%%%%%%%%%%%%%%%%%%%%%%%%%%%%%%%
\section{Introduction}

In order to provide a solution for the two pressing matters in the viability of nuclear energy, spent fuel and power plants that no longer operate, a new design is proposed to construct a borehole-design repository at a shut-down nuclear power plant facility. This design will not only make economic use of the shut-down power plant, but also be able to empty the crowded spent fuel storage pools in many reactors.


%%%%%%%%%%%%%%%%%%%%%%%%%%%%%%%%%%%%%%%%%%%%%%%%%%%%%%%%%%%%%%%%%%%%%%%%%%%%%%%%

\subsection{Background}

The benefits of using a borehole design repository of all kinds is that it 
allows for the creation of regional repositories. The geological requirement of 
a borehole design, crystalline basement rocks at 2,000 ~ 5,000 m deep, are 
relatively common in stable continental regions \cite{arnold_research_2012}.  
Also, its spacial requirements are significantly less than that of a geological 
repository, with only 2km long disposal zone for the amount proposed for Yucca 
Mountain \cite{brady_deep_2009}. The cost of a borehole repository system is 
significantly less than that of a geological one. 

Also, one of the bigger costs of borehole design repository is the repacking of 
spent fuel assemblies to a waste canister. Siting a repository at a non-operating power 
plant facility, which already has the basic infrastructure to handle 
radioactive material, will be much more effective than building a new facility, 
in both cost and licensing. A facility with a dry cask storage site will very 
much likely have the capability to repackage the spent fuel shipments to 
borehole disposal casks. [ percent of reactors that meet the 2 criterion]

\subsection{Motivation}

The proposed design, which is a borehole repository sited at a shut-down power
plant that already has a dry cask storage and fuel handling facility, will be compared to a base design, which is to site a borehole repository 
at a similar location to that of Yucca Mountain. The proposed design has its 
advantage in the fact that it is more appealing to the stakeholders of this 
project. The stakeholders that will be examined in this paper are the federal 
government, the state government, the local government, and the utility company 
that owns the plant.

%%%%%%%%%%%%%%%%%%%%%%%%%%%%%%%%%%%%%%%%%%%%%%%%%%%%%%%%%%%%%%%%%%%%%%%%%%%%%%%



\section{Results and Analysis}



Mainly, the proposed case will have a lower cost, lower new infrastructure 
construction, less time (both for construction and licensing), transportation, 
and a friendlier local community. The proposed design will make it 
significantly easier to proceed in a consent-based manner, where communities 
ask for the incentives emerging from the repository to make up for the economic void created by the shut-down of the power plant. 


\section{Acknowledgments}

This material is based upon work supported by ACDIS.

%%%%%%%%%%%%%%%%%%%%%%%%%%%%%%%%%%%%%%%%%%%%%%%%%%%%%%%%%%%%%%%%%%%%%%%%%%%%%%%%
\bibliographystyle{ans}
\bibliography{bibliography}
\end{document}
