\documentclass{anstrans}
%%%%%%%%%%%%%%%%%%%%%%%%%%%%%%%%%%%
\title{Benefits of Siting a Borehole Repository at a Non-operating Nuclear 
Facility}
\author{Jin Whan Bae,$^{*}$ Abraham Lincoln$^{\dagger}$}

\institute{
$^{*}$Dept. of Nuclear Plasma, and Radiological Engineering, University of Illinois at Urbana-Champaign, Urbana, IL
\and
$^{\dagger}$State Capitol Building, Springfield, IL
}

\email{jbae11@illinois.edu}

%%%% packages and definitions (optional)
\usepackage{graphicx} % allows inclusion of graphics
\usepackage{booktabs} % nice rules (thick lines) for tables
\usepackage{microtype} % improves typography for PDF

\newcommand{\SN}{S$_N$}
\renewcommand{\vec}[1]{\bm{#1}} %vector is bold italic
\newcommand{\vd}{\bm{\cdot}} % slightly bold vector dot
\newcommand{\grad}{\vec{\nabla}} % gradient
\newcommand{\ud}{\mathop{}\!\mathrm{d}} % upright derivative symbol

\begin{document}
%%%%%%%%%%%%%%%%%%%%%%%%%%%%%%%%%%%%%%%%%%%%%%%%%%%%%%%%%%%%%%%%%%%%%%%%%%%%%%%%
\section{Introduction}

For decades the nuclear spent fuel problem has been `indefinitely postponed', 
where sustainable solutions are halted by political, social components. Simply 
getting the nation to agree on such a massive undertaking is a challenging 
feat, let alone about a sensitive topic like radioactive waste. The engineering 
marvel of various personnel both in industry and research has allowed the 
commercial power plant facilities to withstand such a stagnation, with dry 
casks and denser pool packings. However, without a permanent repository, the 
spent fuel inventories keep accumulating and cost to manage them ever rising.

On top of such an alarming situation, it is now the time when many power plant 
facilities near the end of their license, if not already. Provided the option, 
and with the dwindling economic advantage of nuclear power, with dropping gas 
prices and subsidized solar and wind, many corporations don't find nuclear 
attractive any more. The cost of nuclear power is ever increasing, with all the 
cost incorporated with the spent fuel problem.[<- maybe revise this part..?]  
Without a repository, even a 
decommissioned facility suffers a sunk cost maintaining the
spent fuel.

The synchronized combination of the two unfortunate events is threatening the 
survival of nuclear energy, and instead of pouring money into another giant 
project that may or may not succeed, it is time to play smart.

Merging the two problems, and using crisis as an opportunity, may provide a 
viable solution that can kill two birds in one stone.

Siting a borehole-design repository at a shutdown (not decommissioned, for the 
license will be used) power plant facility will not only make economic use of 
the shutdown power plant, but also be able to empty the crowded spent fuel 
storage pools in many reactors.

%%%%%%%%%%%%%%%%%%%%%%%%%%%%%%%%%%%%%%%%%%%%%%%%%%%%%%%%%%%%%%%%%%%%%%%%%%%%%%%%

\subsection{Background}

The benefits of using a borehole design repository of all kinds is that it 
allows for the creation of regional repositories. The geological requirement of 
a borehole design, crystalline basement rocks at 2,000 ~ 5,000m deep, are 
relatively common in stable continental regions \cite{arnold_research_2012}.  
Also, its spacial requirements are significantly less than that of a geological 
repository, with only 2km long disposal zone for the amount proposed for Yucca 
Mountain \cite{brady_deep_2009}. The cost of a borehole repository system is 
signifcantly less than that of a geological one. 

Also, one of the bigger costs of borehole design repository is the repacking of 
spent fuel assemblies to a waste canister. Converting a non-operating power 
plant facility, which already has the basic infrastructure to handle 
radioactive material, will be much more effective than building a new facility, 
in both cost and licensing. A facility with a dry cask storage site will very 
much likely have the capability to repackage the spent fuel shipments to 
borehole disposal casks. [ percent of reactors that meet the 2 criterion]

\subsection{Motivation}

The proposed design, which is a borehole repository sited at a shut-down power
plant that already has a dry cask storage and fuel handling facility, costs 
significantly less than the base design, which is to site a borehole repository 
at a similar location to that of Yucca Mountain. The proposed design has its 
advantage in the fact that it is more appealing to the stakeholders of this 
project. The stakeholders that will be examined in this paper are the federal 
government, the state government, the local government, and the utility company 
that owns the plant.
[ Mainly, the proposed case will have a lower cost, lower new infrastructure 
construction, less time (both for construction and licensing), transportation, 
and a friendlier local community. The proposed design will make it 
significantly easier to proceed in a consent-based manner, where communities 
ask for the incentives emerging from the repository. ] 

In this paper, the proposed design is compared to the base case in various 
economical, political, and social dimensions. Also, the most efficient process 
of the project is discussed, stemming from past researches. 
%%%%%%%%%%%%%%%%%%%%%%%%%%%%%%%%%%%%%%%%%%%%%%%%%%%%%%%%%%%%%%%%%%%%%%%%%%%%%%%%





\section{Results and Analysis}



The results suggest that siting a borehole repository at a shut down power 
plant is beneficial in multiple aspects. The incorporation of a borehole 
repository creates economic opportunities like using of an otherwise sunk-cost 
of a non-operating power 
plant facility. Also, it eases the process of constructing and operating a 
repository facility with its already-built infrastructure. Lastly, it carries 
most value in the fact that it solves the two problems that nuclear power 
faces, with one simple solution.

\section{Acknowledgments}

This material is based upon work supported by ACDIS.

%%%%%%%%%%%%%%%%%%%%%%%%%%%%%%%%%%%%%%%%%%%%%%%%%%%%%%%%%%%%%%%%%%%%%%%%%%%%%%%%
\bibliographystyle{ans}
\bibliography{bibliography}
\end{document}
