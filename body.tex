%%%%%%%%%%%%%%%%%%%%%%%%%%%%%%%%%%%%%%%%%%%%%%%%%%%%%%%%%%%%
\section{Case Definition and Methodology}

This paper sets the proposed case to building a 70,000 \gls{MTHM} capacity borehole repository at the Clinton Power Plant in Illinois. The base case is to build a standalone borehole repository at a location similar to that of Yucca Mountain with the same capacity. 

\subsection{Proposed Case Methodology and Definition}
 In order to minimize transport cost, a central location is preferred. An
  elementary analysis on the transportation of spent fuel is done by
   calculating the total amount of waste multiplied by the distance it has to travel
    ( in units of MTHM*km) using the Havershine formula. First, the coordinates of
    each power plant is obtained by scraping public data [wikipedia?]
 
     The distance between each storage site 
    (i.e. reactors and \gls{ISFSI}) is calculated by 
    using the havershine formula on the geographical coordinates of the sites. 
    \\ \\
 \begin{gather*} 
 \Phi_1,\Phi_2= latitude \, values \, in \, radians\\
 \lambda_1,\lambda_2= longitude \, values \, in \, radians\\
 \Delta\lambda,\Delta\Phi = difference \, between\, two \, values\,  in\,  radians\\
 a=sin^2(\Delta\Phi)+cos(\Phi_1)cos(\Phi_2)sin^2(\frac{\Delta\lambda}{2})\\
 c= 2*atan2(\sqrt{a},\sqrt{1-a} )\\
 d=(6,371km)c
 \end{gather*}
 
 This distance value is multiplied by the \gls{MTHM} that needs to be transported.
     The spent fuel inventory data
    is from the GC-859 survey data from the \gls{EIA} %\cite{?????}
    and the \gls{CURIE} website. %\citep{arnold_geological_2012}te{}
     From the list of 74 reactors, several candidates
    with the smallest MTHM*Km value is listed below:
    
    
\begin{table}[h]
\centering

    \caption { Reactors with relatively small MTHM*Km value}
    \scalebox{0.86}{
	\begin{tabular}{l|l|l|l}
	\hline
	Reactor & State & $MTHM*km$ & License Area [$km^2$]  \\ \hline
	Clinton & Illinois &  \textbf{77,352,339} & \textbf{57.87}   \\ \hline
	Peach Bottom & Pennsylvania & 85,563,135 & 2.509   \\ \hline
	Indian Point &   New York & 84,097,374 & .967   \\ \hline
	Dresden & Illinois &  \textbf{77,663,969} & 3.856   \\ \hline
	
	\end{tabular}}
\end {table}


The Clinton Power Plant is chosen as the site for the proposed case due to its
low $MTHM*km^2$ value and substantially large license area.\cite{NRC_Clinton}
 Considering that only
 $30km^2$ is required for all the total \gls{SNF} amount, the licensed area at Clinton
  power plant allows more than  enough space to site a borehole repository, which
   avoids possible conflicts with the community from purchasing and utilizing more
    land. 
  
  The proposed case requires a great amount of cooperation from the utility that owns
  the Clinton power plant, Exelon Corporation. Currently, Exelon has no 
  incentive to cooperate, for they do not pay for storage of spent fuel, due to 
  the 2004 settlement with the Department of Justice.Also, Exelon currently owes
  approximately a billion dollars to the \gls{NWF},
  (gaining interest at U.S. Treasury bond rate) when a repository is completed
  \cite{Ewing_2009}
  Exelon has an incentive to cooperate,
  since it will earn profit throughout the construction and operation of the 
  repository at its power plant, as well as to utilize the unused land mass
  in a lucrative manner. Also, Exelon would not have to pay for operation or 
  construction, since it is the government's responsibility to dispose nuclear
  spent fuel \cite{Ewing_2009}. 
  
  
\subsection{Base Case Methodology and Definition}
The base case is presented in order to demonstrate the cost savings and efficiencies 
that arise from the proposed case. The base case mimics the Yucca Mountain Project
but is a borehole-type repository. Costs include new licensing and processing facility
 for repacking the spent fuel assemblies.


\section {Incentives to Various Stakeholders}

Prior to discussing the benefits of the proposed case over the base case, the list of
 stakeholders and their incentives are listed below, with a number indicating the 
 magnitude of the importance of the incentive.
 
 
\begin{table}[h]

\centering
\caption {Incentive Criterion and Weight for Each Stakeholder}
\scalebox{0.65}{
	\begin{tabular}{l|l|l|l|l|l}
	\hline
	 & Federal & State & Local & Utility & Environmental \\ \hline
	Job Creation &   & 1 & 3 & 1 &   \\ \hline
	Transport[$MTHM*km$] & 2 & 1 & 2 & & 2\\ \hline
	No Need for new treatment license & 2 & & & 1 & \\ \hline
	No Need for additional land purchase & 3 & 2 & 3 & & 2 \\ \hline
	Emptying Spent Fuel Storage Pools & 3 & & & 3 & \\ \hline
	Net Cost & 3 & & & 3 & \\ \hline
	No New Above-Ground Facility Construction & 3 & & & 3 & 2\\ \hline
	
	\end{tabular}}
\end{table}

\subsection{Job Creation}

Building a spent nuclear fuel repository is no easy task. It is a task that requires
numerous experts and labourers. Also, operating and maintaining a nuclear power plant
requires numerous experts and labourers. In case of the proposed case, the Clinton
 Power Station has approximately 700 employees living in nearby counties with an
additional several hundred contractors during fuel outages.\cite{Exelon}
The existing skilled workers and local talent for maintenance, transport and catering
services can be utilized without bringing a whole new group of workers to the area. \cite{IAEA_2008}. 

The base case does produce more jobs, since it needs additional constructions such
as the repackaging infrastructure. It is estimated that jobs created during the
construction was between 3,200 to 4,000 \cite{Riddel_2003}.
 However, the job creation may not be as
appreciated greatly by the local community than that of the proposed case.

Additionally, an estimate by the Illinois State University on fracking the New Albany
Shale in southern Illinois estimated that such a project can create 1,000-47,000 jobs
\cite{Loomis_2012}. Translating the workforce to central Illinois and the borehole
project should create somewhere in the low and medium estimate, which is about 10,000
jobs.

%%something about the state?


Thus, the void created by the shutdown of the Clinton plant can be, though not
completely, filled by the new construction of a borehole repository. The construction
will prioritize local hires as an incentive to ease local opposition on repository
 siting. Employment during the operation of Yucca Mountain was estimated to range from
 2,000 to 5,000 jobs, \cite{Riddel_2003} which means that the borehole repository
 would at leaste require half of the workforce for the same capacity.  

\subsection{Transport}
Transport of radioactive material is a difficult matter, and poses one of the greatest
problems in siting a repository. The proposed case, according to the crude analysis,
has the least amount of required transportation of spent fuel. Also, it is
conveniently located near the Canadian National rail line \cite{waleed_regional_2015}. 

%%% NWPA allows only "Transshipment of spent nuclear fuel to another civilian 
%%% reactor within the same utility system" [Title I, Subtitle B, Sec. 134(a)]



Conversely, siting the base case will have a $km*MTHM$ value of $209,575,157 km*MTHM$, 
which is approximately 2.7 times more than that of the proposed case. Also, %%railways?

\subsection{No Need for a New Treatment License}
%NRC license application

\subsection{No Need for Additional Land Purchase}
The proposed case has a licensed land area of approximately $58km^2$ and $20km^2$
 cooling heat sink, the Clinton Lake, with only $.6km^2$ being used for the facility.
  \cite{NRC_Clinton} This leaves enough room left for a 70,000 \gls{MTHM} borehole
  repository without additional land purchase from the public. 


\subsection{Emptying Spent Fuel Storage Pools }


\subsection{Net Cost}
The proposed case has a larger 

\subsection{No New Above-Ground Facility Construction}
The proposed case, being a once-operating nuclear power plant, has the facility to 
repack the spent fuel assemblies into a disposal cask. However, this facility needs 
to be upgraded to handle a large influx of spent fuel assemblies, and should be
preferably automatic, to minimize worker exposure. The transported spent fuel
assemblies are repacked and inspected at the upgraded facility, and is sent to the
emplacement tubes for final disposal. Not having to build a new above-ground facility
should greatly increase the public perception, for it seems like there's minimal
impact.

The base case requires a new above-ground facility, which not only costs a great
amount, but also will be considered problematic to the public's eye. 
